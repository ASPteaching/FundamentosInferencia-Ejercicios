% Options for packages loaded elsewhere
\PassOptionsToPackage{unicode}{hyperref}
\PassOptionsToPackage{hyphens}{url}
%
\documentclass[
]{article}
\usepackage{amsmath,amssymb}
\usepackage{iftex}
\ifPDFTeX
  \usepackage[T1]{fontenc}
  \usepackage[utf8]{inputenc}
  \usepackage{textcomp} % provide euro and other symbols
\else % if luatex or xetex
  \usepackage{unicode-math} % this also loads fontspec
  \defaultfontfeatures{Scale=MatchLowercase}
  \defaultfontfeatures[\rmfamily]{Ligatures=TeX,Scale=1}
\fi
\usepackage{lmodern}
\ifPDFTeX\else
  % xetex/luatex font selection
\fi
% Use upquote if available, for straight quotes in verbatim environments
\IfFileExists{upquote.sty}{\usepackage{upquote}}{}
\IfFileExists{microtype.sty}{% use microtype if available
  \usepackage[]{microtype}
  \UseMicrotypeSet[protrusion]{basicmath} % disable protrusion for tt fonts
}{}
\makeatletter
\@ifundefined{KOMAClassName}{% if non-KOMA class
  \IfFileExists{parskip.sty}{%
    \usepackage{parskip}
  }{% else
    \setlength{\parindent}{0pt}
    \setlength{\parskip}{6pt plus 2pt minus 1pt}}
}{% if KOMA class
  \KOMAoptions{parskip=half}}
\makeatother
\usepackage{xcolor}
\usepackage[margin=1in]{geometry}
\usepackage{longtable,booktabs,array}
\usepackage{calc} % for calculating minipage widths
% Correct order of tables after \paragraph or \subparagraph
\usepackage{etoolbox}
\makeatletter
\patchcmd\longtable{\par}{\if@noskipsec\mbox{}\fi\par}{}{}
\makeatother
% Allow footnotes in longtable head/foot
\IfFileExists{footnotehyper.sty}{\usepackage{footnotehyper}}{\usepackage{footnote}}
\makesavenoteenv{longtable}
\usepackage{graphicx}
\makeatletter
\def\maxwidth{\ifdim\Gin@nat@width>\linewidth\linewidth\else\Gin@nat@width\fi}
\def\maxheight{\ifdim\Gin@nat@height>\textheight\textheight\else\Gin@nat@height\fi}
\makeatother
% Scale images if necessary, so that they will not overflow the page
% margins by default, and it is still possible to overwrite the defaults
% using explicit options in \includegraphics[width, height, ...]{}
\setkeys{Gin}{width=\maxwidth,height=\maxheight,keepaspectratio}
% Set default figure placement to htbp
\makeatletter
\def\fps@figure{htbp}
\makeatother
\setlength{\emergencystretch}{3em} % prevent overfull lines
\providecommand{\tightlist}{%
  \setlength{\itemsep}{0pt}\setlength{\parskip}{0pt}}
\setcounter{secnumdepth}{5}
\ifLuaTeX
  \usepackage{selnolig}  % disable illegal ligatures
\fi
\usepackage{bookmark}
\IfFileExists{xurl.sty}{\usepackage{xurl}}{} % add URL line breaks if available
\urlstyle{same}
\hypersetup{
  hidelinks,
  pdfcreator={LaTeX via pandoc}}

\author{}
\date{\vspace{-2.5em}}

\begin{document}

{
\setcounter{tocdepth}{2}
\tableofcontents
}
\section*{Presentación}\label{presentaciuxf3n}
\addcontentsline{toc}{section}{Presentación}

\subsection*{Objetivo}\label{objetivo}
\addcontentsline{toc}{subsection}{Objetivo}

El objetivo de estos ejercicios es proporcionar unos materiales de soporte para la asignatura de ``Inferencia Estadística'' del \href{https://www.uoc.edu/es/estudios/masters/master-universitario-bioinformatica-bioestadistica}{Máster interuniversitario de Bioiestadística y Bioinformática} impartido conjuntamente por la \href{https://www.uoc.edu}{Universitat Oberta de Catalunya (UOC)} y la \href{https://www.ub.edu}{Universidad de Barcelona (UB)}.

Esta asignatura adolece de las características habituales de las asignaturas de posgrado, y especialmente de un posgrado de estadística (y bioinformática), que muestran algunas de las cosas que no debe de ser esta asignatura:

Tal como se indica en la introducción a las notas de soporte del curso, este debería:

\begin{itemize}
\tightlist
\item
  Servir para repasar y consolidar los conceptos básicos que la mayoría de estudiantes traerán consigo.
\item
  Además, y sobretodo, debe proporcionar una visión general, lo más completa posible dentro de las limitaciones de tiempo, del campo de la inferencia estadística
\end{itemize}

Y, naturalmente, una de las formas de consolidar conocimientos, como en cualquier disciplina cuantitatva,es a traves de la resolución de ejercicios que permiten reflexionar, comprender y ver como se aplican los conceptos teóricos introducidos.

Para ello, estos materiales contienen una serie de ejercicios similares a los que se proponen en las actividades y pruebas de evaluación continua de la asignatura.

La mayoría de los ejercicios estan resueltos, pero \emph{es importante intentar resolverlos de forma autónoma antes de consultar la solución}.

En general los ejercicios no presuponen ningún conocimiento especial de matemáticas, más allá de las habilidades básicas que se adquieren durante los estudios de una carrera de ciencias o de ingeniería.

\end{document}
